% Created 2019-12-08 Sun 16:47
% Intended LaTeX compiler: pdflatex
\documentclass[11pt]{article}
\usepackage[utf8]{inputenc}
\usepackage[T1]{fontenc}
\usepackage{graphicx}
\usepackage{grffile}
\usepackage{longtable}
\usepackage{wrapfig}
\usepackage{rotating}
\usepackage[normalem]{ulem}
\usepackage{amsmath}
\usepackage{textcomp}
\usepackage{amssymb}
\usepackage{capt-of}
\usepackage{hyperref}
\author{Alice Ip, Kexin Liu, Lily Lau, Meijing Li}
\date{December 08, 2019}
\title{CS/SE 3GC3 - Computer Graphics Final Project}
\hypersetup{
 pdfauthor={Alice Ip, Kexin Liu, Lily Lau, Meijing Li},
 pdftitle={CS/SE 3GC3 - Computer Graphics Final Project},
 pdfkeywords={},
 pdfsubject={},
 pdfcreator={Emacs 26.3 (Org mode 9.1.9)}, 
 pdflang={English}}
\begin{document}

\maketitle

\section*{Project Description}
\label{sec:org1d8940a}
A cooking game that takes place in a 3-D simulating kitchen,
 where the player selects a recipe and must interact with
 ingredient objects and tool objects and follow the instructions
 given to create the dish.

\section*{Keyboard Commands}
\label{sec:orgd56cdd5}
\begin{itemize}
\item q - quit

\item up arrow - move camera to look up
\item down arrow - move camera to look down
\item left arrow - move camera to look left
\item right arrow - move camerat to look right

\item left-click to select an ingredient or tool
\item right click to use tool (if a tool is selcted)
\end{itemize}


\section*{Features from Prototype}
\label{sec:orgf8f58d8}
The appropriate obj files for the ingredients/tools/textures
were obtained from several sources (that are cited in the
readme file), and edited as appropriate for our needs. 

Summed Up:
\begin{itemize}
\item Basic room with floors, walls, lighting
\item Basic meshes for each food object and tool object, and kitchen counter
\item .obj file parser (modification of parser taken from tutorial)
\begin{itemize}
\item Extracts data (mesh, texture, normals, faces) into c++ vectors
\end{itemize}
\item Loading object meshes into the room
\end{itemize}

\section*{New Features}
\label{sec:orgac3d105}

\subsection*{Key Features (Advanced Graphics Features)}
\label{sec:org701e791}
\begin{itemize}
\item Lighting [5\%]
\item Textures [10\%] 
\begin{itemize}
\item On objects, wall, floor
\end{itemize}
\item Ray Casting [10\%]
\begin{itemize}
\item To select tool/ingredient objects in screen
\end{itemize}
\item Non-geometric primitives(bitmaps, pixel maps) [10\%]
\begin{itemize}
\item Pixel map for choosing a recipe, and End screen
\item Interaction handlers on the pixel map
\item Timer text on the top left corner (Bitmap String)
\end{itemize}
\item\relax [5\%]
\end{itemize}

\subsection*{Other Relevant Features}
\label{sec:orgc7815aa}
Some of the features that were implemented in the final implementation include the
 application of textures  on the objects. Depending on which recipe
is chosen by the user, the appropriate instructions for the recipe and only the 
necessary ingredients for that object are loaded in. The recipe instructions are 
loaded in a separate window.  Once a recipe is chosen, a 
timer is displayed in the window indicating remaining time for that recipe. The
scoring system is based on the amount of time left when the user completes the 
recipe. In order to perform actions on the ingredients, we implemented ray casting
 to detect collision between the mouse position (left click) and objects in the screen.
 If there is an intersection,
then the selected object will increase in size (to indicate that it was selected) and
can be moved to a new position (by moving the mouse to a new position) and deselected
 (using left-click). A tool can be selected and used on a ingredient, by right-clicking 
(when selected) over a ingredient. The appropriate mesh (cut version of ingrediant, or
cooked version of ingredient) will appear. When the correct combinations of ingredients
 are put together, the game will end.

Summed up:
\begin{itemize}
\item Basic Camera rotation
\item Movement of 3D objects on 2D plane
\item Timer
\item Recipe/Control instructions in new window
\item Start screen
\item End screen
\item Use of both orthographic and perspective projection
\item Ray shading
\end{itemize}

\subsection*{Recipe Details}
\label{sec:orgc125921}

\begin{itemize}
\item Curry
\begin{itemize}
\item Objects: Potato, Tomato, Onion, Knife, Pot
\item Use knife on Potato (whole)
\item Use Potato (cut) on Pot
\item Use knife on Tomato (whole)
\item Use Tomato (cut) on Pot
\item Use knife on Onion (whole)
\item Use Onion (cut) on Pot
\item Curry Complete
\end{itemize}

\item Fruit Salad
\begin{itemize}
\item Objects: Mango, Orange, Banana, Knife
\item Use Knife on Mango (whole)
\item Use Mango (cut) on Bowl
\item Use Knife on Orange (whole)
\item Use Orange (cut) on Bowl
\item Use Knife on Banana (whole)
\item Use Banana (cut) on Bowl
\item Fruit Salad Complete
\end{itemize}

\item Steak
\begin{itemize}
\item Objects: Steak, Pan
\item Use Steak on Pan
\item Wait 10 Seconds
\item Steak Complete
\end{itemize}
\end{itemize}
\end{document}
